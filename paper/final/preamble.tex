\usepackage{amsmath,amssymb,graphicx,bbm,xcolor,latexsym}
\usepackage{xspace}
\usepackage{verbatim}

\usepackage{pgfplots} 
% and optionally (as of Pgfplots 1.3): 
\pgfplotsset{compat=newest} 
\pgfplotsset{plot coordinates/math parser=false} 
\newlength\figureheight 
\newlength\figurewidth 

% CLEVER REFERENCES
% This package automatically infers the type of
% references (e.g. equations, figures, theorems, etc.) using the
% \cref* family of commands. If it is to be used, it should be loaded
% last. If hyperref and varioref is also used, the order should be
% varioref, hyperref, cleveref. To use with theorems correctly, either
% amsthm or ntheorem should be loaded, and all \newtheorem definitions
% must be placed AFTER the cleveref package is loaded. To use with a
% language other than English, the language should be set globally in
% the \documentclass command rather than solely in babel.

\usepackage[capitalise]{cleveref} % Options: capitalise, noabbrev, nameinlink

% If using cleveref, all \newtheorem commands should be placed after this line.
\newtheorem{theorem}{Theorem}
\newtheorem{lemma}{Lemma}
\newtheorem{corollary}{Corollary}
\newtheorem{proposition}{Proposition}
\newtheorem{assumption}{Assumption}

\newtheorem{definition}{Definition}
\newtheorem{problem}{Problem}

\newtheorem{example}{Example}
\newtheorem{remark}{Remark}
\newtheorem{notation}{Notation}
\newtheorem{term}{Terminology}


%%%%%%%%%% Start TeXmacs macros
\newcommand{\cdummy}{\cdot}
\newcommand{\comma}{{,}}
\newcommand{\mathd}{\mathrm{d}}
\newcommand{\mathe}{\mathrm{e}}
\newcommand{\nospace}{}
\newcommand{\tmaffiliation}[1]{\thanks{\textit{Affiliation:} #1}}
\newcommand{\tmop}[1]{\ensuremath{\operatorname{#1}}}
%%%%%%%%%% End TeXmacs macros

\newcommand{\simplenote}[1]{{\color{red}#1}}
\newcommand{\simpletodo}[1]{\simplenote{TODO: #1}}
\usepackage{todonotes}

\newcommand{\RHnote}[1]{\simplenote{RH: #1}}
\newcommand{\TNnote}[1]{\simplenote{TN: #1}}

\newcommand{\eg}{e.g.,\xspace}
\newcommand{\ie}{i.e.,\xspace}
\newcommand{\etc}{etc.\xspace}
\newcommand{\cf}{cf.~}

% The \overbar command, similar to \bar and \overline. It scales with the content as does \overline, but it's a bit shorter than \overline. It's actually \overline shortened by 1.5mu in each side
\newcommand*{\overbar}[1]{\mkern 2.0mu\overline{\mkern-2.0mu#1\mkern-1.0mu}\mkern 1.0mu}
