\section{Conclusions}
\label{sec:conclusions}

We developed an approach which combines techniques from control theory and computer science to solve the energy scheduling problem for large-scale systems with fast sampling time.
The approach uses a model abstraction method based on the concept of simulation relations between transition systems, which allows us to reduce the original multi-state multi-binary-input model to a single-state single-real-input model with input bound tracking and safety guarantees.
Unlike the mixed integer programming approach, which has computational issues for any system of practical size, our approach is much more scalable.
While the offline design process may require significant computing power, the run-time implementation is lightweight that it can potentially be implemented on embedded computers.
We validated our approach using Matlab simulations of a room heating system.
Our numerical simulations showed that while the mixed integer programming approach may not work beyond a trivially small-scale system, our approach can handle efficiently a much larger system with a fast sampling rate and a long control horizon.


A future extension of this work is to test the control algorithms on an embedded platform to verify its run-time scalability.
Another direction is to develop better scheduling algorithms for the low-level actuator scheduler, such that not only the averaged interval demand but also the instantaneous demand are reduced.
We also aim to apply the approach to more practical systems other than the room heating system.

%%% Local Variables:
%%% mode: latex
%%% TeX-master: "emsoft15gs"
%%% End:
