\section{Green Scheduling Problem}
\label{sec:gs-problem}

\simplenote{This section can be placed inside the Introduction.}

We consider a variant of the original green scheduling problem where the peak
demand is determined based on the interval averaged total demand.
Specifically, time is divided into equal intervals: $[0, T], [T, 2 T], \ldots$
In each interval $[kT, (k + 1) T]$ the averaged total demand is calculated as $d_k = \frac{E_k}{T}$ where $E_k$ is the total energy consumption during interval $k$.
This is the practical way to determine energy demands, for example in electricity bills and in demand response (DR) programs.
The peak demand, for example to be used in calculating the demand charge, is the maximum interval demand over a given billing period: $\max_k d_k$.
Our goal is to schedule the subsystems to minimize the peak interval demand while maintaining a safe operation.
At the same time, it is desirable to schedule the subsystems in each interval so that the peak instantaneous total demand is also reduced, although this will not affect the demand charge.

As in our previous work, we consider a continuous-time linear system $\Sigma_b$ of the form
%
\begin{equation*}
(\Sigma_b) \qquad \dot{x} = Ax + Bu + Ew
\end{equation*}
%
where $x \in \mathbbm{R}^{n}$ is the state vector, $u \in \{ 0, 1 \}^{m}$ are the binary control inputs that represent the schedules, and $w \in \mathbbm{R}^{p}$ is the disturbance vector.
The safety condition requires that the state $x (t)$ stays in a safe set $\mathcal{X} \subset \mathbbm{R}^n$ at all time.
The disturbances are constrained in a known set $w (t) \in \mathcal{W}$ for all $t$.
Both $\mathcal{X}$ and $\mathcal{W}$ are polyhedral.

At any time, the total demand is defined as a linear combination of the binary
control inputs: $d (t) = \rho^T u (t)$ where $\rho$ is the vector of the individual
power demand of each subsystem. In each time interval $[kT, (k + 1) T]$ the
total energy consumption is $E_k = \int_{kT}^{(k + 1) T} d (t) \mathd t$ and
the averaged total demand is $d_k = \frac{E_k}{T}$. 
In the green scheduling problem, we aim to schedule
the subsystems, \ie to compute the control inputs $u$, so that the energy cost over a finite horizon is minimized.
The energy cost consists of a charge for the peak demand and possibly a charge for the energy
consumtion.
Specifically, we want to minimize the following cost function defined over a horizon of $N$ time intervals:
\begin{equation}
  \label{eq:cost-function} \text{cost} = c_d \cdot \max_{0 \leqslant k
  \leqslant N - 1} d_k + \textstyle\sum_{0 \leqslant k \leqslant N - 1} c_{e, k} \cdot
  E_k
\end{equation}
where $c_d$ is the fixed price for the peak demand and $c_{e, k}$ is the
time-varying price for the interval energy consumption.
Typically $c_d \gg c_{e, k}$ which gives customers incentive to reduce their peak demands.


The green scheduling problem can be formulated as minimizing the above cost function subject to the constraints:
\begin{align*}
  %\mathrm{minimize}_{u (\cdummy)} &  & c_d \cdot \max_{0 \leqslant k \leqslant N - 1} d_k + \sum_{0 \leqslant k \leqslant N - 1} c_{e, k} \cdot E_k\\
  %\text{subject to} &
  & \dot{x} (t) = Ax (t) + Bu (t) + Ew (t)\\%, \, d (t) = \rho^T u (t)\\%, \qquad \forall t\\
  & x (t) \in \mathcal{X}, \quad u(t) \in \{0,1\}^{m}, \quad w (t) \in \mathcal{W}\\%, \qquad \forall t\\
  & E_k = \int_{kT}^{(k + 1) T} \rho^T u (\tau) \mathd \tau, \, d_k = \frac{E_k}{T}%, \qquad \forall k = 0, \ldots, N - 1
\end{align*}
 for all $t$ and all $k = 0, \ldots, N - 1$.
This optimization is intractable because $u (\cdummy)$ is infinite
dimensional.
If we discretize $\Sigma_b$ with sampling time $T$ and assume that $u (t)$ and $w (t)$ are constant in each time interval, the optimization becomes:
\begin{align}
  \operatorname*{minimize}_{u_{0}, \ldots, u_{N - 1}} \quad & c_d \cdot \max_{0
  \leqslant k \leqslant N - 1} d_k + \textstyle\sum_{0 \leqslant k \leqslant N - 1}
  c_{e, k} \cdot E_k  \label{eq:MILP}\\
  \text{subject to} \quad & x_{k+1} = A_{T} x_{k} + B_{T} u_{k} + E_{T} w_{k}  \nonumber\\
  & d_k = \rho^T u_{k} \nonumber\\
  & x_{k} \in \mathcal{X}, \quad u_{k} \in \{0,1\}^{m}, \quad w_{k} \in \mathcal{W} \nonumber\\
  & E_k = T d_k \nonumber
\end{align}
where the constraints hold for all $k = 0, \ldots, N - 1$.
Here, the subscript $k$ denotes the value of a variable at time step $k$, and matrices $A_{T}$, $B_{T}$, and $E_{T}$ are of the discrete-time dynamical model.
Note that the safety condition, that $x_k$ stays inside $\XSet$ at all $k$, must be robust to the unknown but bounded disturbances $w$, and thereby results in conservative control inputs trading off performance for safety.
In practice, disturbance forecasts are usually available, for example in the forms of weather forecast and occupancy schedules, which should be exploited to obtain more accurate predictions of future states and hence less conservative control inputs and better performance.
Therefore, the disturbances are modeled as $w_{k} = \tilde{w}_{k} + \delta_{k}$ where $\tilde{w}_{k} \in \WSet$ is the forecast and $\delta_{k}$ is the forecast uncertainty.
The forecast accuracy, as a bounded set of $\delta_{k}$, is assumed to be known and certainly smaller than $\WSet$.
The optimization \eqref{eq:MILP} is a mixed-integer linear program (MILP) and can be solved by an MILP solver.
However, except for small-scale systems with only a few control inputs and a short horizon $N$, the number of binary variables can be prohibitively large and the MILP is difficult to solve.
For example, if $\Sigma_b$ has $m=40$\simpletodo{May need to update this number!!!} control inputs with sampling time $T = 5 \text{min}$ for a horizon of $N = 288$ steps in $24$ hours, the MILP will have $11 \comma 520$ binary variables, not to mention the continuous state and disturbance variables.
Solving such large MILPs often requires powerful computers with commercial optimization
solvers.
Moreover, due to the uncertainty caused by the disturbances, the control decisions need to be adjusted regularly by solving the optimization \eqref{eq:MILP} repeatedly at every time step (model predictive control (MPC)).
Therefore, for any practical size of the system, implementing this approach can be highly demanding, if even possible, in terms of run-time hardware and software requirements.
Although there are techniques in MPC to reduce the complexity, \eg move blocking, an MILP solver is still required.
Certainly the controller cannot be implemented on an embedded processor with limited processing power and memory.


Our goal in this paper is to develop a scalable green scheduling algorithm that can potentially run on embedded processors, even for systems of moderate practical size.
We consider a slightly more general problem than \eqref{eq:MILP}:
\begin{align}
  \operatorname{minimize}_{u (\cdummy)} \quad & c_d \cdot \max_{0 \leqslant k \leqslant
  N - 1} d_k + \textstyle\sum_{0 \leqslant k \leqslant N - 1} c_{e, k} \cdot E_k 
  \label{eq:GS-optim}\\
  \text{subject to} \quad
  & \dot{x} (t) = Ax (t) + Bu (t) + Ew (t) \quad \forall t \geq 0 \nonumber\\
  & u(t) \in \{0,1\}^{m} \quad \forall t \geq 0 \nonumber\\
  & x_{k} = x (kT) \in \XSet \nonumber\\
  & w (t) = w_{k} \in \mathcal{W} \quad \forall t \in [kT, (k + 1) T) \nonumber\\
  & E_k = \int_{kT}^{(k + 1) T} \rho^{T} u(\tau) \mathd \tau, \quad d_k = \frac{E_k}{T} \nonumber
\end{align}
where the constraints hold for all $k = 0, \ldots, N - 1$.
Problem~\eqref{eq:GS-optim} is more general than \eqref{eq:MILP} because we consider a
continuous-time schedule $u (\cdummy)$, which is not necessarily constant
during each time interval.



%%% Local Variables:
%%% mode: latex
%%% TeX-master: "emsoft15gs"
%%% End:
