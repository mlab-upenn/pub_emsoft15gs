\section{Model Abstraction for Green\\Scheduling}
\label{sec:abstraction-gs}

\simpletodo{Relation R could be exact but for flexibility, any state x that approximates
the aggregated s, so we use the error band.}

Going back to the green scheduling problem, we apply the framework of input-constrained simulation relations to abstract the high-dimensional model $\Sigma_{a}$ by a low-dimensional model $\Sigma_{s}$.
We then derive a feedback control law that allows $\Sigma_{a}$ to track the state of $\Sigma_{s}$ while maintaining their simulation relation, all their state and input constraints, as well as the input upper-bounds set by $\Sigma_{s}$.
In particular, we consider the scalar discrete-time system $\Sigma_{s}$:
\begin{align}
  (\Sigma_s) \quad
  & s_{k + 1} = \alpha s_k + v_k + \beta^T w_{k}   \label{eq:abstract-system} \\
  & s_k \in \SSet \subseteq \mathbbm{R},
  v_k \in \VSet \subseteq \mathbbm{R},
  (s_k, v_k) \in \Omega \subseteq \SSet \times \VSet \nonumber
 \end{align}
%
The intuition is that $s_{k}$ represents an aggregated state of all the subsystems at time step $k$ and $v_{k}$ specifies an upper-bound of the total energy demand.
For example, consider an energy system consisting of $n$ rooms and a heater in each room, where the heaters have fixed heating powers and can only be switched on and off.
At time $k$, $s_{k}$ can represent the total enthalpy of all rooms, while $v_{k}$ bounds the total heating energy of all heaters during that interval.
Because the system is subject to heat loss and disturbances, the dynamics of $s$ have the form of \eqref{eq:abstract-system}.

The abstraction of $\Sigma_{a}$, \ie the input-constrained simulation relation $\RSet$, depends on how the control input $v_{k}$ is computed and how the disturbances are represented in $\Sigma_{s}$.
In practice, forecasts of the disturbances, denoted $\tilde{w}_{k}$, with known bounded accuracies are often available and used in computing the control inputs for both $\Sigma_{s}$ and $\Sigma_{a}$.
The forecast accuracies are represented by a vector $\zeta \geq 0$ in $\mathbb{R}^{p}$ such that the error between $w_{k}$ and $\tilde{w}_{k}$ are always bounded element-wise as $-\zeta \leqslant w_{k} - \tilde{w}_{k} \leqslant \zeta$ for all $k$.

We consider two control approaches in this paper:
\begin{enumerate}
\item \textbf{Feedforward approach:} The control sequence $v_{0}$, $v_{1}$, \dots, $v_{N-1}$ of $\Sigma_{s}$ are computed using the disturbance forecasts $\tilde{w}_{0}, \tilde{w}_{1}, \dots, \tilde{w}_{N-1}$ as nominal disturbances for the horizon.  These control inputs are not adjusted until the end of the horizon; in other words, the control law for $\Sigma_{s}$ is feedforward.  Consequently, the trajectory of $\Sigma_{s}$ is nominal without taking into account actual disturbances.  The control input $\eta_{k}$ for $\Sigma_{a}$ is computed based on the actual state $x_{k}$ of $\Sigma_{a}$ and the nominal state, control, and disturbances of $\Sigma_{s}$, hence it is a feedback law.
\item \textbf{Feedback approach:} At each time step $k$, $v_{k}$ is decided using the actual state $s_{k}$, which is computed from the observed actual state $x_{k}$. %, and the disturbance forecast $\tilde{w}_{k}$.
  The state trajectory of $\Sigma_{s}$ is therefore actual, not nominal.  The control input $\eta_{k}$ is calculated from the actual states $s_{k}$ and $x_{k}$, and $v_{k}$ and $\tilde{w}_{k}$.  Hence, both control laws for $\Sigma_{a}$ and $\Sigma_{s}$ are feedback.
\end{enumerate}

%Assuming that the parameters $\alpha$ and $\beta$ of $\Sigma_{s}$ are given, 
We derive the abstraction $\Sigma_{s}$ of $\Sigma_{a}$ for each approach in three steps:
\begin{enumerate}
\item $\Sigma_{s}$ and $\Sigma_{a}$ are formulated as transition systems $T_{1}$ and $T_{2}$ respectively;
\item Assuming a certain form of $\mathcal{R}$, we derive %a necessary and sufficient
  conditions for $\mathcal{R}$ to be an input-constrained simulation relation;
\item Based on the conditions, we develop algorithms to compute the parameters of $\Sigma_{s}$, its joint constraint $\Omega$, and the simulation relation $\mathcal{R}$.
\end{enumerate}
%
%In \cref{sec:abstraction-gs:sigma-s} we will discuss how the parameters of $\Sigma_{s}$ are computed.


\subsection{Feedforward Approach}
\label{sec:abstraction-gs:feedforward}

\subsubsection{Transition Systems}
\label{sec:abstraction-gs:feedforward:Ts}

In the feedforward approach, the evolution of $\Sigma_{s}$ is nominal, where the (nominal) disturbances are known and deterministic.
We formulate $T_{1}$ as follows:
\begin{itemize}
\item State $q_{1} \equiv s$ and state set $Q_{1} = \mathcal{S}$.
\item Control label $\upsilon_{1} \equiv v$ with $\mathcal{U}_{1} = \mathcal{V}$.
\item Disturbance label $\delta_{1} \equiv \tilde{w}$ is the disturbance forecast with $\mathcal{D}_{1} = \mathcal{W}$ being the set of possible disturbances.
\item Transition set $\rightarrow_{1} = \{ (s, v, \tilde{w}, s^{+}) \,|\, (s,v) \in \Omega,\allowbreak \tilde{w} \in \WSet, %\mathcal{D}_{1},
  \allowbreak s^{+} = \alpha s + v + \beta^{T} \tilde{w} \in \SSet\}$. %Q_{1}
\end{itemize}
%
The transition system $T_{2}$ of $\Sigma_{a}$ is defined as:
\begin{itemize}
\item State is the averaged state $q_{2} \equiv \overbar{x}$ with $Q_{2} = \mathcal{X}$.
\item Control label $\upsilon_{2} \equiv \eta$ with $\mathcal{U}_{2} = [0,1]^{m}$.
\item Disturbance label: $\Sigma_{a}$ %the averaged system 
  is influenced by the actual disturbances $w_{k}$, therefore $\delta_{2} \equiv w$ with $\mathcal{D}_{2} = \mathcal{W}$.
\item Transition set: recall that at every time step $k+1$, $\overbar{x}_{k+1}$ is reset to $x((k+1)T)$, which is within a bounded error $e_{k+1}$ from $A_{T} \overbar{x}_{k} + B_{T} \eta_k + E_{T} w_{k}$ (\cf \cref{thm:averaged-sys-bound}).  This reset can be modeled as a non-deterministic but bounded perturbation of the state.  Therefore we define $\rightarrow_{2} = \{ (\overbar{x}, \eta, w, \overbar{x}^{+}) \,|\, \overbar{x} \in \XSet, \eta \in \mathcal{U}_{2}, w \in \mathcal{W},\allowbreak \overbar{x}^{+} \in \XSet,\allowbreak \overbar{x}^{+} = A_{T} \overbar{x}+ B_{T} \eta + E_{T} w + e,\allowbreak 
\underline{\varepsilon} \leqslant V (e + \tilde{B} \eta) \leqslant \overbar{\varepsilon}\}$.
\end{itemize}
%
Because $v_{k}$ specifies an upper-bound of the total energy demand $T \rho^{T} \eta_{k}$, we define the joint input constraint $\mathcal{R}_{u} = \{ (v, \eta) \in \mathcal{U}_{1} \times \mathcal{U}_{2} \,|\, T \rho^{T} \eta \leq v \}$.
Finally, the forecast accuracy between nominal $\tilde{w}$ of $\Sigma_{s}$ and actual $w$ of $\Sigma_{a}$ is modeled by the disturbance relation $\mathcal{R}_{d} = \{ (\tilde{w}, w) \in \mathcal{D}_{1} \times \mathcal{D}_{2} \,|\, -\zeta \leqslant w - \tilde{w} \leqslant \zeta \}$.


\subsubsection{Conditions for \RSet} % Input-constrained Simulation Relation \protect\RSet}
\label{sec:abstraction-gs:feedforward:conditions}

To derive the model abstraction of $\Sigma_{a}$, the input-constrained simulation relation $\RSet$ must be computed.
Specifically, we need to be able to certify if any given relation $\RSet$ of $s$ and $\overbar{x}$ is an input-constrained simulation relation of $T_{1}$ by $T_{2}$.
The following theorem provides such a necessary and sufficient condition for $\RSet$.
Here, the product between a matrix $M \in \mathbb{R}^{m \times n}$ and a set $\PSet \subseteq \mathbb{R}^{n}$ is defined as the set $M \PSet = \{ y \in \mathbb{R}^{m} \,|\, y = M x, x \in \PSet\}$.
The identity matrix is denoted by $\IdentityMatrix$, whose dimensions are often omitted when they are obvious from the context.

\begin{theorem}
  \label{thm:abstraction-gs:necessary-sufficient}
  A set $\RSet \subseteq \SSet \times \XSet$ is an input-constrained simulation relation of $T_{1}$ by $T_{2}$ if and only if
  \begin{align*}
    M_l \PSet_l \subseteq M_r \PSet_r
  \end{align*}
  where 
  \begin{gather*}
    M_l =
          \begin{bmatrix}
            0 & 0 & 1 & 0 & 0 & 0 \\
            \alpha & 0 & 1 & 0 & \beta^{T} & 0 \\
            0 & A_T & 0 & E_T & E_T & \IdentityMatrix
          \end{bmatrix}\\
    M_r =
      \begin{bmatrix}
        0 & 0 & 1 & 0 \\
        1 & 0 & 0 & 0 \\
        0 & \IdentityMatrix & 0 & \tilde{B} - B_{T}
      \end{bmatrix}
  \end{gather*}
  and
  \begin{align*}
    \PSet_l =& \big\{ (s,\overbar{x}, v, \delta, \tilde{w}, \varepsilon) \,\big|\,
              (s,\overbar{x}) \in \RSet,\ (s,v) \in \Omega, -\zeta \leqslant \delta \leqslant \zeta,\big.\\
            &\qquad \big. \tilde{w} \in \WSet,  \alpha s + v + \beta^{T} \tilde{w} \in \SSet, \underline{\varepsilon} \leqslant \varepsilon \leqslant \overbar{\varepsilon} \big\} \\
    \PSet_r =& \big\{(s^+, \overbar{x}^{+}, \overbar{v}, \eta) \,\big|\,
               \overbar{x}^{+} \in \XSet, (s^{+}, \overbar{x}^{+}) \in \RSet, \eta \in [0,1]^{m},\big.\\
             &\qquad \big. T \rho^{T} \eta \leqslant \overbar{v}\big\}
  \end{align*}
\end{theorem}
The proof can be found in \cref{sec:proof:abstraction-gs:necessary-sufficient}.

\subsection{Feedback Approach}
\label{sec:abstraction-gs:feedback}


% \subsection{Parameters of \protect$\Sigma_{s}$}
% \label{sec:abstraction-gs:sigma-s}




%%% Local Variables:
%%% mode: latex
%%% TeX-master: "emsoft15gs"
%%% End:
