\section{Model Abstraction for Green\\Scheduling}
\label{sec:abstraction-gs}

Going back to the green scheduling problem, we apply the framework of input-constrained simulation relations to abstract the high-dimensional model $\Sigma_{a}$ by a low-dimensional model $\Sigma_{s}$.
We then derive a feedback control law that allows $\Sigma_{a}$ to track the state of $\Sigma_{s}$ while maintaining their simulation relation, all their state and input constraints, as well as the input upper-bounds set by $\Sigma_{s}$.
In particular, we consider the scalar discrete-time system $\Sigma_{s}$:
\begin{align}
  (\Sigma_s) \quad
  & s_{k + 1} = \alpha s_k + v_k + \beta^T w_{k}   \label{eq:abstract-system} \\
  & s_k \in \SSet \subseteq \mathbbm{R},
  v_k \in \VSet \subseteq \mathbbm{R},
  (s_k, v_k) \in \Omega \subseteq \SSet \times \VSet \nonumber
 \end{align}
%
The intuition is that $s_{k}$ represents an aggregated state of all the subsystems at time step $k$ and $v_{k}$ specifies an upper-bound of the total energy demand.
For example, consider an energy system consisting of $n$ rooms and a heater in each room, where the heaters have fixed heating powers and can only be switched on and off.
At time $k$, $s_{k}$ can represent the total enthalpy of all rooms, while $v_{k}$ bounds the total heating energy of all heaters during that interval.
Because the system is subject to heat loss and disturbances, the dynamics of $s$ have the form of \eqref{eq:abstract-system}.

The abstraction of $\Sigma_{a}$, \ie the input-constrained simulation relation $\RSet$, depends on how the control input $v_{k}$ is computed and how the disturbances are represented in $\Sigma_{s}$.
In practice, forecasts of the disturbances, denoted $\tilde{w}_{k}$, with known bounded accuracies are often available and used in computing the control inputs for both $\Sigma_{s}$ and $\Sigma_{a}$.
The forecast accuracies are represented by a vector $\zeta \geq 0$ in $\mathbb{R}^{p}$ such that the error between $w_{k}$ and $\tilde{w}_{k}$ are always bounded element-wise as $-\zeta \leqslant w_{k} - \tilde{w}_{k} \leqslant \zeta$ for all $k$.

We consider two control approaches in this paper:
\begin{enumerate}
\item \textbf{Feedforward approach:} The control sequence $v_{0}$, $v_{1}$, \dots, $v_{N-1}$ of $\Sigma_{s}$ are computed using the disturbance forecasts $\tilde{w}_{0}, \tilde{w}_{1}, \dots, \tilde{w}_{N-1}$ as nominal disturbances for the horizon.  These control inputs are not adjusted until the end of the horizon; in other words, the control law for $\Sigma_{s}$ is feedforward.  Consequently, the trajectory of $\Sigma_{s}$ is nominal without taking into account actual disturbances.  The control input $\eta_{k}$ for $\Sigma_{a}$ is computed based on the actual state $x_{k}$ of $\Sigma_{a}$ and the nominal state, control, and disturbances of $\Sigma_{s}$, hence it is a feedback law.
\item \textbf{Feedback approach:} At each time step $k$, $v_{k}$ is decided using the actual state $s_{k}$, which is computed from the observed actual state $x_{k}$. %, and the disturbance forecast $\tilde{w}_{k}$.
  The state trajectory of $\Sigma_{s}$ is therefore actual, not nominal.  The control input $\eta_{k}$ is calculated from the actual states $s_{k}$ and $x_{k}$, and $v_{k}$ and $\tilde{w}_{k}$.  Hence, both control laws for $\Sigma_{a}$ and $\Sigma_{s}$ are feedback.
\end{enumerate}

%Assuming that the parameters $\alpha$ and $\beta$ of $\Sigma_{s}$ are given, 
We derive the abstraction $\Sigma_{s}$ of $\Sigma_{a}$ for each approach in three steps:
\begin{enumerate}
\item $\Sigma_{s}$ and $\Sigma_{a}$ are formulated as transition systems $T_{1}$ and $T_{2}$ respectively;
\item Assuming a certain form of $\RSet$, we derive %a necessary and sufficient
  conditions for $\RSet$ to be an input-constrained simulation relation;
\item Based on the conditions, we develop algorithms to compute the parameters of $\Sigma_{s}$, its joint constraint $\Omega$, and the simulation relation $\RSet$.
\end{enumerate}
%
%In \cref{sec:abstraction-gs:sigma-s} we will discuss how the parameters of $\Sigma_{s}$ are computed.


\subsection{Feedforward Approach}
\label{sec:abstraction-gs:feedforward}

\subsubsection{Transition Systems}
\label{sec:abstraction-gs:feedforward:Ts}

In the feedforward approach, the evolution of $\Sigma_{s}$ is nominal, where the (nominal) disturbances are known and deterministic.
We formulate $T_{1}$ as follows:
\begin{itemize}
\item State $q_{1} \equiv s$ and state set $Q_{1} = \mathcal{S}$.
\item Control label $\upsilon_{1} \equiv v$ with $\mathcal{U}_{1} = \mathcal{V}$.
\item Disturbance label $\delta_{1} \equiv \tilde{w}$ is the disturbance forecast with $\mathcal{D}_{1} = \WSet$ being the set of possible disturbances.
\item Transition set $\rightarrow_{1} = \{ (s, v, \tilde{w}, s^{+}) \,|\, (s,v) \in \Omega,\allowbreak \tilde{w} \in \WSet, %\mathcal{D}_{1},
  \allowbreak s^{+} = \alpha s + v + \beta^{T} \tilde{w} \in \SSet\}$. %Q_{1}
\end{itemize}
%
The transition system $T_{2}$ of $\Sigma_{a}$ is defined as:
\begin{itemize}
\item State is the averaged state $q_{2} \equiv \overbar{x}$ with $Q_{2} = \XSet$.
\item Control label $\upsilon_{2} \equiv \eta$ with $\mathcal{U}_{2} = [0,1]^{m}$.
\item Disturbance label: $\Sigma_{a}$ %the averaged system 
  is influenced by the actual disturbances $w_{k}$, therefore $\delta_{2} \equiv w$ with $\mathcal{D}_{2} = \WSet$.
\item Transition set: recall that at every time step $k+1$, $\overbar{x}_{k+1}$ is reset to $x((k+1)T)$, which is within a bounded error $e_{k+1}$ from $A_{T} \overbar{x}_{k} + B_{T} \eta_k + E_{T} w_{k}$ (\cf \cref{thm:averaged-sys-bound}).  This reset can be modeled as a non-deterministic but bounded perturbation of the state.  Therefore we define $\rightarrow_{2} = \{ (\overbar{x}, \eta, w, \overbar{x}^{+}) \,|\, \overbar{x} \in \XSet, \eta \in \mathcal{U}_{2}, w \in \WSet,\allowbreak \overbar{x}^{+} = A_{T} \overbar{x}+ B_{T} \eta + E_{T} w + e \in \XSet,\allowbreak 
\underline{\varepsilon} \leqslant V (e + \tilde{B} \eta) \leqslant \overbar{\varepsilon}\}$.
\end{itemize}
%
Because $v_{k}$ specifies an upper-bound of the total energy demand $T \rho^{T} \eta_{k}$, we define the joint input constraint $\RSet_{u} = \{ (v, \eta) \in \mathcal{U}_{1} \times \mathcal{U}_{2} \,|\, T \rho^{T} \eta \leq v \}$.
Finally, the forecast accuracy between nominal $\tilde{w}$ of $\Sigma_{s}$ and actual $w$ of $\Sigma_{a}$ is modeled by the disturbance relation $\RSet_{d} = \{ (\tilde{w}, w) \in \mathcal{D}_{1} \times \mathcal{D}_{2} \,|\, -\zeta \leqslant w - \tilde{w} \leqslant \zeta \}$.


\subsubsection{Conditions for \RSet} % Input-constrained Simulation Relation \protect\RSet}
\label{sec:abstraction-gs:feedforward:conditions}

To derive the model abstraction of $\Sigma_{a}$, the input-constrained simulation relation $\RSet$ must be computed.
Specifically, we need to be able to certify if any given relation $\RSet$ of $s$ and $\overbar{x}$ is an input-constrained simulation relation of $T_{1}$ by $T_{2}$.
The following theorem provides such a necessary and sufficient condition for $\RSet$.
Here, the product between a matrix $M \in \mathbb{R}^{m \times n}$ and a set $\PSet \subseteq \mathbb{R}^{n}$ is defined as the image of $\PSet$ under the linear map $M$, \ie $M \PSet = \{ y \in \mathbb{R}^{m} \,|\, y = M x, x \in \PSet\}$.
The identity matrix is denoted by $\IdentityMatrix$, whose dimensions are often omitted when they are obvious from the context.
The symbol $\Zero$ denotes a zero vector or matrix.

\begin{theorem}
  \label{thm:abstraction-gs:necessary-sufficient}
  A set $\RSet \subseteq \SSet \times \XSet$ is an input-constrained simulation relation of $T_{1}$ by $T_{2}$ for the feedforward approach if and only if
  \begin{align}
    \label{eq:abstraction-gs:necessary-sufficient}
    M_l \PSet_l \subseteq M_r \PSet_r
  \end{align}
  where 
  \begin{gather*}
    M_l =
          \begin{bmatrix}
            0 & \Zero & 1 & \Zero & \Zero & \Zero \\
            \alpha & \Zero & 1 & \Zero & \beta^{T} & \Zero \\
            0 & A_T & 0 & E_T & E_T & \IdentityMatrix
          \end{bmatrix}\\
    M_r =
      \begin{bmatrix}
        0 & \Zero & 1 & \Zero \\
        1 & \Zero & 0 & \Zero \\
        0 & \IdentityMatrix & 0 & \tilde{B} - B_{T}
      \end{bmatrix}
  \end{gather*}
  and
  \begin{align*}
    \PSet_l =& \big\{ (s,\overbar{x}, v, \delta, \tilde{w}, \varepsilon) \,\big|\,
              (s,\overbar{x}) \in \RSet,\ (s,v) \in \Omega, -\zeta \leqslant \delta \leqslant \zeta,\big.\\
            &\qquad \big. \tilde{w} \in \WSet, \tilde{w} + \delta \in \WSet, \alpha s + v + \beta^{T} \tilde{w} \in \SSet, \underline{\varepsilon} \leqslant \varepsilon \leqslant \overbar{\varepsilon} \big\} \\
    \PSet_r =& \big\{(s^+, \overbar{x}^{+}, \overbar{v}, \eta) \,\big|\,
               \overbar{x}^{+} \in \XSet, (s^{+}, \overbar{x}^{+}) \in \RSet, \eta \in [0,1]^{m},\big.\\
             &\qquad \big. T \rho^{T} \eta \leqslant \overbar{v}\big\}
  \end{align*}
\end{theorem}
The proof can be found in \cref{sec:proof:abstraction-gs:necessary-sufficient}.

Although \cref{thm:abstraction-gs:necessary-sufficient} is useful for verifying whether a given $\RSet$ is an input-constrained simulation relation, it does not allow us to directly compute $\RSet$ since $\RSet$ appears on both sides of \eqref{eq:abstraction-gs:necessary-sufficient}.
Furthermore, computing the image of a set under a linear map and checking for a set inclusion are generally difficult.
Therefore, we will impose a certain structure on the relation $\RSet$ that results in polyhedral sets $\PSet_{l}$ and $\PSet_{r}$.
We will also derive a simpler condition for $\RSet$, which can be verified by checking the feasibility of a linear program.

As mentioned in the overview in \cref{sec:overview} and again at the beginning of this section, $s$ represents an aggregated state of all the subsystems, for example the total enthalpy of an energy system.
Thus, we can choose $s$ to be approximately a linear combination of all the (energy) states of the subsystems.
Let $s \approx c^{T} x$, where $c$ is a constant vector in $\mathbb{R}^{n}$.
The set $\RSet$ is chosen to represent this relationship between $s$ and $x$ as 
\begin{equation}
  \label{eq:abstraction-gs:R}
  \RSet = \{ (s,x) \in \SSet \times \XSet \,|\, | c^{T} x - s | \leqslant \gamma \} \text,
\end{equation}
where $\gamma \geqslant 0$ is a design parameter.
Intuitively, in an energy system, the set of $x \in \XSet$ such that $(s,x) \in \RSet$, for any $s \in \SSet$, is the set of all states that have roughly the same aggregated energy level $s$ (up to an error of $\gamma$).

We have assumed in \cref{sec:gs-problem} that $\XSet$ and $\WSet$ are polyhedra.
Suppose also that $\Omega$ is a polyhedron.
Let their hyperplane representations be
\begin{gather*}
  \XSet = \{ x \in \mathbb{R}^{n} \,|\, H_{x} x \leqslant k_{x} \}, \qquad
  \WSet = \{ w \in \mathbb{R}^{p} \,|\, H_{w} x \leqslant k_{w} \}\\
  \Omega = \{ (s,v) \in \mathbb{R}^{2} \,|\, H_{\Omega}^{s} s + H_{\Omega}^{v} v \leqslant k_{\Omega} \}
\end{gather*}
where $H_{x}$, $k_{x}$, $H_{w}$, $k_{w}$, $H_{\Omega}^{s}$, $H_{\Omega}^{v}$, $k_{\Omega}$ are matrices and vectors of appropriate dimensions.
Because $s$ and $v$ are scalars, we can bound them as $\underline{s} \leqslant s \leqslant \overbar{s}$ and $\underline{v} \leqslant v \leqslant \overbar{v}$, where $\underline{s}$, $\overbar{s}$, $\underline{v}$, $\overbar{v}$ are respectively the lower- and upper-bounds of $s$ and $v$.
With these assumptions, $\PSet_{l}$ and $\PSet_{r}$ become polyhedral sets.
We can now state a sufficient condition for $\RSet$ in the next theorem.

\begin{theorem}
  \label{thm:abstraction-gs:sufficient-LP}
  A set $\RSet$ as defined in \eqref{eq:abstraction-gs:R} is an input-constrained simulation relation of $T_{1}$ by $T_{2}$ for the feedforward approach if there exist a real matrix $Z$ of non-negative elements, a real matrix $Q$, and a real vector $q$ such that
  \begin{subequations}
    \label{eq:abstraction-gs:sufficient-LP}
    \begin{align}
      M_{r} Q &= \IdentityMatrix \\
      M_{r} q &= \Zero \\
      Z H_{l} &= H_{r} Q M_{l} \\
      Z k_{l} &\leqslant k_{r} - H_{r} q
    \end{align}
  \end{subequations}
  in which
  \begin{gather*}
    H_{l} =
    \begin{bmatrix}
      1 \\ %  & \Zero  &  \Zero
      -1 \\ % & \Zero  &  \Zero
        & H_{x} \\ % &  \Zero
      -1  & c^{T} \\ % &  \Zero
      1   & -c^{T} \\ % &  \Zero
      H_{\Omega}^{s} & & H_{\Omega}^{v} \\
      & & 1 \\
      & & -1 \\
      & & & & H_{w} \\
      & & & H_{w} & H_{w} \\
      & & & \IdentityMatrix \\
      & & & -\IdentityMatrix \\
      \alpha & & 1 & & \beta^{T} \\
      -\alpha & & -1 & & -\beta^{T} \\
      & & & & & V \\
      & & & & & -V
    \end{bmatrix}, \;
    k_{l} =
    \begin{bmatrix}
      \overbar{s} \\
      -\underline{s} \\
      k_{x} \\
      \gamma \\
      \gamma \\
      k_{\Omega} \\
      \overbar{v} \\
      -\underline{v} \\
      k_{w} \\
      k_{w} \\
      \zeta \\
      \zeta \\
      \overbar{s} \\
      -\underline{s} \\
      \overbar{\varepsilon} \\
      -\underline{\varepsilon}
    \end{bmatrix} \\
    %
    H_{r} =
    \begin{bmatrix}
      & H_{x} \\ % &  \Zero
      -1  & c^{T} \\ % &  \Zero
      1   & -c^{T} \\ % &  \Zero
      & & -1 & T \rho^{T} \\
      & & & \IdentityMatrix \\
      & & & -\IdentityMatrix
    \end{bmatrix}, \;
    k_{r} =
    \begin{bmatrix}
      k_{x} \\
      \gamma \\
      \gamma \\
      0 \\
      \One \\
      \Zero
    \end{bmatrix} \text.
  \end{gather*}
\end{theorem}
In the above matrices and vectors, the blank spaces are zero blocks and $\One$ denotes vector of 1's.
The theorem is proved in \cref{sec:proofs:abstraction-gs:sufficient-LP}.

Although \cref{thm:abstraction-gs:sufficient-LP} is only sufficient, it reduces the condition to checking the feasibility of the linear program~\eqref{eq:abstraction-gs:sufficient-LP}, which can be solved rather efficiently by any LP solver such as Gurobi, MOSEK, CPLEX, CLP.

We note that the design parameters, which are ($\alpha$, $\beta$, $c$, $\gamma$, $H_{\Omega}^{s}$, $H_{\Omega}^{v}$, $k_{\Omega}$), are multiplied by the elements of $Z$ in \eqref{eq:abstraction-gs:sufficient-LP}.
Consequently, if we want to solve \eqref{eq:abstraction-gs:sufficient-LP} to also find the design parameters, the problem becomes a bilinear program, which is intractable to solve for any significant size.
We will discuss a way to compute these parameters in \cref{sec:abstraction-gs:computation}, after we derive similar results for the feedback control approach in the next section.

\subsection{Feedback Approach}
\label{sec:abstraction-gs:feedback}

\subsubsection{Transition Systems}
\label{sec:abstraction-gs:feedback:Ts}

In the feedback approach, the evolution of $\Sigma_{s}$ is not nominal as it is influenced by the actual disturbances, which are unknown but within known accuracies from their forecasts.
To model this phenomenon, we will include the disturbance forecasts in the control labels of both $T_{1}$ and $T_{2}$, while their disturbance labels are the errors between the forecasts and the actual disturbances.
Specifically, we define $T_{1}$ as follows:
\begin{itemize}
\item State $q_{1} \equiv s$ and state set $Q_{1} = \mathcal{S}$.
\item Control label $\upsilon_{1} \equiv (v, \tilde{w}_{1})$ with $\mathcal{U}_{1} = \VSet \times \WSet$.
\item Disturbance label $\delta_{1}$ is the disturbance forecast error $\delta_{1} = w - \tilde{w}_{1}$; $\mathcal{D}_{1} = \{ \delta_{1} \in \mathbb{R}^{p} \,|\, -\zeta \leqslant \delta_{1} \leqslant \zeta\}$.
\item Transition set $\rightarrow_{1} = \{ (s, v, \tilde{w}_{1}, \delta_{1}, s^{+}) \,|\, (s,v) \in \Omega,\allowbreak \tilde{w}_{1} \in \WSet,\allowbreak \delta_{1} \in \mathcal{D}_{1},\allowbreak \tilde{w}_{1} + \delta_{1} \in \WSet,\allowbreak s^{+} = \alpha s + v + \beta^{T} (\tilde{w}_{1} + \delta_{1}) \in \SSet\}$. %Q_{1}
\end{itemize}
%
The transition system $T_{2}$ of $\Sigma_{a}$ is defined as:
\begin{itemize}
\item State $q_{2} \equiv \overbar{x}$; $Q_{2} = \XSet$.
\item Control label $\upsilon_{2} \equiv (\eta, \tilde{w}_{2})$ with $\mathcal{U}_{2} = [0,1]^{m} \times \WSet$.
\item Disturbance label $\delta_{2}$ is also the forecast error; $\mathcal{D}_{2} = \{ \delta_{2} \in \mathbb{R}^{p} \,|\, -\zeta \leqslant \delta_{2} \leqslant \zeta\}$.
\item Transition set: we again model the state reset as a non-deterministic but bounded perturbation of $\overbar{x}$.   Define $\rightarrow_{2} = \{ (\overbar{x}, \eta, \tilde{w}_{2}, \delta_{2}, \overbar{x}^{+}) \,|\, \overbar{x} \in \XSet, \eta \in \mathcal{U}_{2}, \tilde{w}_{2} \in \WSet,\allowbreak \delta_{2} \in \mathcal{D}_{2},\allowbreak \tilde{w}_{2} + \delta_{2} \in \WSet,\allowbreak \overbar{x}^{+} = A_{T} \overbar{x}+ B_{T} \eta + E_{T} (\tilde{w}_{2} + \delta_{2}) + e \in \XSet,\allowbreak \underline{\varepsilon} \leqslant V (e + \tilde{B} \eta) \leqslant \overbar{\varepsilon}\}$.
\end{itemize}
%
The joint input constraint $\RSet_{u}$ needs to take into account that the disturbance forecasts are the same for $T_{1}$ and $T_2$, hence $\RSet_{u} = \{ ((v, \tilde{w}_{1}), (\eta, \tilde{w}_{2})) \in \mathcal{U}_{1} \times \mathcal{U}_{2} \,|\, T \rho^{T} \eta \leq v, \tilde{w}_{1} = \tilde{w}_{2} \}$.
Finally, because $T_{1}$ and $T_{2}$ are subject to the same actual disturbances, we define the disturbance relation $\RSet_{d} = \{ (\delta_{1}, \delta_{2}) \in \mathcal{D}_{1} \times \mathcal{D}_{2} \,|\, \delta_{1} = \delta_{2} \}$.

\subsubsection{Conditions for \RSet} % Input-constrained Simulation Relation \protect\RSet}
\label{sec:abstraction-gs:feedback:conditions}

The only differences between the feedback and feedforward approaches are that in the feedback approach: (a) the control input $v$ for $T_{1}$ must robustly drive $s$ to $\SSet$ regardless of the forecast error $\delta$; (b) $s$ is updated with the actual disturbance $\tilde{w} + \delta$.
Therefore, the necessary and sufficient condition in \cref{thm:abstraction-gs:necessary-sufficient} still holds for the feedback approach with modified matrix $M_{l}$ and set $\PSet_{l}$:
\begin{gather*}
  M_l =
  \begin{bmatrix}
    0 & \Zero & 1 & \Zero & \Zero & \Zero \\
    \alpha & \Zero & 1 & \beta^{T} & \beta^{T} & \Zero \\
    0 & A_T & 0 & E_T & E_T & \IdentityMatrix
  \end{bmatrix}\\
  \begin{aligned}
    \PSet_l =& \big\{ (s,\overbar{x}, v, \delta, \tilde{w}, \varepsilon) \,\big|\,
    (s,\overbar{x}) \in \RSet,\ (s,v) \in \Omega, -\zeta \leqslant \delta \leqslant \zeta,\big.\\
    &\qquad \big. \tilde{w} \in \WSet, \tilde{w} + \delta \in \WSet, \alpha s + v + \beta^{T} \tilde{w} \in \SSet \ominus \beta^{T} \mathcal{D}_{1},\\
    &\qquad \underline{\varepsilon} \leqslant \varepsilon \leqslant \overbar{\varepsilon} \big\}
  \end{aligned}
\end{gather*}
where $\ominus$ denotes the Pontryagin difference between two sets.

  \begin{gather*}
    H_{l} =
    \begin{bmatrix}
      1 \\ %  & \Zero  &  \Zero
      -1 \\ % & \Zero  &  \Zero
        & H_{x} \\ % &  \Zero
      -1  & c^{T} \\ % &  \Zero
      1   & -c^{T} \\ % &  \Zero
      H_{\Omega}^{s} & & H_{\Omega}^{v} \\
      & & 1 \\
      & & -1 \\
      & & & & H_{w} \\
      & & & \IdentityMatrix \\
      & & & -\IdentityMatrix \\
      \alpha & & 1 & & \beta^{T} \\
      -\alpha & & -1 & & -\beta^{T} \\
      & & & & & V \\
      & & & & & -V
    \end{bmatrix}, \;
    k_{l} =
    \begin{bmatrix}
      \overbar{s} \\
      -\underline{s} \\
      k_{x} \\
      \gamma \\
      \gamma \\
      k_{\Omega} \\
      \overbar{v} \\
      -\underline{v} \\
      k_{w} \\
      \zeta \\
      \zeta \\
      \overbar{s} \\
      -\underline{s} \\
      \overbar{\varepsilon} \\
      -\underline{\varepsilon}
    \end{bmatrix} \\
    %
    H_{r} =
    \begin{bmatrix}
      & H_{x} \\ % &  \Zero
      -1  & c^{T} \\ % &  \Zero
      1   & -c^{T} \\ % &  \Zero
      & & -1 & T \rho^{T} \\
      & & & \IdentityMatrix \\
      & & & -\IdentityMatrix
    \end{bmatrix}, \;
    k_{r} =
    \begin{bmatrix}
      k_{x} \\
      \gamma \\
      \gamma \\
      0 \\
      \One \\
      \Zero
    \end{bmatrix} \text.
  \end{gather*}


Although \cref{thm:abstraction-gs:sufficient-LP} is only sufficient, it reduces the condition to checking the feasibility of the linear program~\eqref{eq:abstraction-gs:sufficient-LP}, which can be solved rather efficiently by any LP solver such as Gurobi, MOSEK, CPLEX, CLP.


\subsection{Computation of Design Parameters}
\label{sec:abstraction-gs:computation}



% \subsection{Parameters of \protect$\Sigma_{s}$}
% \label{sec:abstraction-gs:sigma-s}




%%% Local Variables:
%%% mode: latex
%%% TeX-master: "emsoft15gs"
%%% End:
